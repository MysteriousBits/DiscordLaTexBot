\documentclass[varwidth, margin=2mm]{standalone}

\usepackage{amsmath}
% \usepackage{mathtools}
\usepackage{amsthm, amssymb}
% \usepackage{graphicx}
% \usepackage{xcolor}
% \usepackage{tikz, tikz-cd}

\begin{document}

Lets call each m people a \emph{Group}, their common friend the \emph{Target} and a group including $P$ a \emph{Good Group}. Suppose $P$ has $k$ friends.
\\\\
\textbf{Claim:} $P$ has $m$ friends or $k=m$.
\\\\
\textbf{Lemma 1:} \emph{Each \emph{Good Group} has a distinct \emph{Target}}
\\
Suppose two \emph{Good Groups} $G_1$ and $G_2$ have same target $T$. Now make a \emph{Group} $G_3$ from the members of $G_1$ and $G_2$ excluding $P$. Now all the members of $G_3$ have 2 \emph{Targets} $T$ and $P$,  which is a contradiction.
\\\\ 
\textbf{Proof:} Suppose, $k > m$. There are $\binom{k}{m-1}$ different \emph{Targets} of all the \emph{Good Groups}  according to Lemma 1. Now all the \emph{Targets} are among those $k$ people as they are friends of $P$. Then $\binom{k}{m-1} \leq k$. But its a contradiction. 
\\
Hence $k=m$. \qed

\end{document}